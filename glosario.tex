%% ---------------------------------------------------------------------------
%% paNotation.tex
%%
%% Notation
%%
%% $Id: paNotation.tex,v 1.15 2004/03/30 05:55:59 alvarado Exp $
%% ---------------------------------------------------------------------------

\cleardoublepage
\renewcommand{\nomname}{Lista de símbolos y abreviaciones}
\markboth{\nomname}{\nomname}
\renewcommand{\nompreamble}{\addcontentsline{toc}{chapter}{\nomname}%
\setlength{\nomitemsep}{-\parsep}
\setlength{\itemsep}{10ex}
}

%%
% Símbolos en la notación general
% (es posible poner la declaración en el texto
%%

%\symg[t]{$\sys{\cdot}$}{Transformación realizada por un sistema}
%\symg[yscalar]{$y$}{Escalar.}
%\symg[zconjugado]{$\conj{z}$}{Complejo conjugado de $z$}
%\symg[rcomplexreal]{$\Re(z)$ o $z_{\Re}$}{Parte real del número complejo $z$}
%\symg[icompleximag]{$\Im(z)$ o $z_{\Im}$}{Parte imaginaria del número
%                                        complejo $z$}
%\symg[jimaginario]{$j$}{$j=\sqrt{-1}$}
%\symg[C]{$\setC$}{Conjunto de los números complejos.}

%%
% Algunas abreviaciones
%%

%\syma{DAS}{Sistemas de Asistencia al Conductor}
\syma{ADAS}{Sistemas Avanzados de Asistencia al Conductor}
%\syma{CADAS}{Sistemas Colaborativos Avanzados de Asistencia al Conductor}
\syma{AC}{Computación Aproximada}
\syma{AV}{Visión Artificial}
\syma{CV}{Visión por Computadora}
\syma{AD}{Conducción Autónoma}
\syma{KIT}{Instituto Tecnológico de Karlsruhe}
\syma{TEC}{Tecnológico de Costa Rica}
\syma{R\&D}{Investigación y Desarrollo}
\syma{ROI}{Región de Interés}



\printnomenclature[20mm]

